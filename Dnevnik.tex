\documentclass{report}
\usepackage[utf8]{inputenc}
\usepackage[slovene]{babel}
\usepackage{makeidx}
\usepackage{hyperref}
\usepackage{booktabs}
\usepackage{textgreek}
\usepackage{multirow} % for spanning rows
\usepackage{changepage}
\usepackage{ulem}
\usepackage{afterpage}
\usepackage{float}
\usepackage{bm}
\usepackage{mathtools}
\usepackage{amsmath}
\usepackage{graphicx}
\graphicspath{ {./images/} }

\hypersetup{
  colorlinks=true,         % Colored links instead of boxes
  linkcolor=blue,          % Color for internal links
  citecolor=green,         % Color for citations
  filecolor=magenta,       % Color for file links
  urlcolor=cyan,           % Color for external links
  bookmarksopen=true,      % Display bookmarks when the PDF is opened
  bookmarksnumbered=true,  % Display section numbers in bookmarks
  bookmarksdepth=2,        % Set the depth of bookmarks (1 for chapters, 2 for sections, etc.)
}

\title{Vaje iz mehanike}
\author{Matija Zanjkovič\thanks{Mentor: Marko Šterk}}
\date{Maribor, 2023}

\begin{document}

\maketitle
\thispagestyle{empty}

\null\newpage

\thispagestyle{empty}
\tableofcontents

\listoftables

\listoffigures

\thispagestyle{empty}

\clearpage  % Start a new page



\setcounter{page}{1}  % Set the page number to 1

\chapter{Vaja 1: Merjenje gostote}
\section{Gostota trdne snovi}
\subsection{Naloga 1}
Z merjenjem dimenzij (širine (\bm{a}), višine (\bm{b}), dolžine (\bm{c})) in mase (\bm{m}) kvadra določite gostoto \bm{(\rho)} snovi, iz katere je narejen kvader. 
Gostoto izračunajte po enačbi \bm{\rho\ =\ m/V}, kjer je \bm{V} prostornina (\bm{V\ =\ abc}). Določite tudi napako gostote snovi.
\subsection{Sistematične napake merilnikov}
Napaka kljunastega merila: \bm{0.05\ mm}\\
Napaka mikrometra: \bm{0.01\ mm}\\
Napaka tehtnice: \bm{0.1\ g}\\

\subsection{Meritve}

% Tabela za meritev a
\begin{table}[H]
  \centering
  \caption{Meritve dimenzije a}
  \begin{adjustwidth}{-2 cm}{0 cm}
  \begin{tabular}{cccccccc}
  \toprule
  Meritev & $a_{izm}\ [mm]$ & $\overline{a}\ [mm]$ & $a_{izm} - \overline{a}\ [mm]$ & $\Delta a_{sist}\ [mm]$ & $\sigma [mm]$&  $\Delta a_{sl}\ [mm]$ & $a\ [mm]$\\
  \midrule
  1 & 8.16 & \multirow{10}{*}{8.16} & 0 & \multirow{10}{*}{0.01} & \multirow{10}{*}{0.02} & \multirow{10}{*}{0.01} \\
  2 & 8.15 &  & -0.01 &\\
  3 & 8.20 &  & \sout{0.04} &\\
  4 & 8.18 &  & 0.02 &\\
  5 & 8.16 &  & 0.00 & & & & 8.16 \ \pm \ 0.02\\
  6 & 8.15 &  & -0.01 & & & & = \\
  7 & 8.16 &  & 0.00 & & & & 8.16 $\cdot$ (1 \ \pm \ 0.002)\\
  8 & 8.17 &  & 0.01 &\\
  9 & 8.10 &  & \sout{-0.06} &\\
  10 & 8.12 &  & \sout{-0.04} &\\
  % Add more rows here
  \bottomrule
  \end{tabular}
  \end{adjustwidth}
\end{table}

% Tabela za meritev b
\begin{table}[H]
  \centering
  \caption{Meritve dimenzije b}
  \begin{adjustwidth}{-2 cm}{0 cm}
  \begin{tabular}{cccccccc}
  \toprule
  Meritev & $b_{izm}\ [mm]$ & $\overline{b}\ [mm]$ & $b_{izm} - \overline{b}\ [mm]$ & $\Delta b_{sist}\ [mm]$ & $\sigma\ [mm]$ & $\Delta b_{sl}\ [mm]$ &  $b\ [mm]$\\
  \midrule
  1 & 25.25 & \multirow{10}{*}{25.23} & 0.02 & \multirow{10}{*}{0.05} & \multirow{10}{*}{0.03} & \multirow{10}{*}{0.01}\\
  2 & 25.20 &  & -0.03 &\\
  3 & 25.20 &  & -0.03 &\\
  4 & 25.25 &  & 0.02 &\\
  5 & 25.25 &  & 0.02 & & & & 25.23 \ \pm \ 0.06\\
  6 & 25.20 &  & \sout{-0.03} & & & & =\\
  7 & 25.20 &  & \sout{-0.03} & & & & 25.23 $\cdot$ (1 \ \pm \ 0.002)\\
  8 & 25.20 &  & \sout{-0.03} &\\
  9 & 25.25 &  & 0.02 &\\
  10 & 25.25 &  & 0.02 &\\
  % Add more rows here
  \bottomrule
  \end{tabular}
  \end{adjustwidth}
\end{table}

% Tabela za meritev c
\begin{table}[H]
  \centering
  \caption{Meritve dimenzije c}
  \begin{adjustwidth}{-2 cm}{0 cm}
  \begin{tabular}{cccccccc}
  \toprule
  Meritev & $c_{izm}\ [mm]$ & $\overline{c}\ [mm]$ & $c_{izm} - \overline{c}\ [mm]$ & $\Delta c_{sist}\ [mm]$ & \sigma & $\Delta c_{sl}\ [mm]$ & $c\ [mm]$\\
  \midrule
  1 & 40.00 & \multirow{10}{*}{40.02} & -0.02 & \multirow{10}{*}{0.05} & \multirow{10}{*}{0.02} & \multirow{10}{*}{0.01} & \multirow[b]{4}{*}{40.02 \ \pm \ 0.06}\\
  2 & 40.00 &  & -0.02 &\\
  3 & 40.10 &  & \sout{0.08} &\\
  4 & 40.00 &  & -0.02 &\\
  5 & 40.00 &  & -0.02 & & & & \multirow{2}{*}{=}\\
  6 & 40.00 &  & -0.02 & & &  \\
  7 & 40.00 &  & -0.02 & & & & \multirow[t]{4}{*}{40.02 $\cdot$ (1 \ \pm \ 0.001)} \\
  8 & 40.05 &  & \sout{0.03} &\\
  9 & 40.05 &  & \sout{0.03} &\\
  10 & 40.00 &  & -0.02 &\\
  % Add more rows here
  \bottomrule
  \end{tabular}
  \end{adjustwidth}
\end{table}

Meritev mase: $m = 22.8 \ \pm \ 0.1\ g \ = \ 22.8 \cdot (1 \ \pm \ 0.004)\ g$

\pagebreak

\subsection{Računanje gostote}

Gostota se računa po enačbi:
\begin{equation}
  \rho = \frac{m}{V}
\end{equation}
\\
Vendar najprej rabimo volumen telesa. Ker gre za kvader lahko uporabimo enačbo:
\begin{equation}
  V = abc
\end{equation}
\\
Tako torej dobimo:

\begin{equation}
  \label{eq:1}
  \begin{gathered}
    V = 8.16 \ (1 \pm 0.002)\ mm \cdot 25.23 \ (1 \pm 0.002)\ mm \cdot 40.02 \ (1 \pm 0.001)\ mm\\
    V = 8.16 \cdot 25.23 \cdot 40.02 \ (1 \pm (0.002 + 0.002 + 0.001))\ mm^3\\
    V = 8240 \ (1 \pm 0.005) \ mm^3
  \end{gathered}
\end{equation}
\\
Sedaj lahko izračunamo gostoto telesa:

\begin{equation}
  \label{eq:1}
  \begin{gathered}
    \rho = \frac{22.8 \cdot (1 \ \pm \ 0.004)\ g}{8240 \cdot (1 \pm 0.005) \ mm^3} \\
    \rho = 2.77 \cdot 10^{-3} \cdot (1 \ \pm \ 0.009) \ \frac{g}{mm^3} \\
    \rho = 2770 \cdot (1 \ \pm \ 0.009) \ \frac{kg}{m^3}
  \end{gathered}
\end{equation}

\subsection{Rezultati}
Prišli smo do rezultata, da je gostota telesa $\rho = 2770 \cdot (1 \ \pm \ 0.009) \ \frac{kg}{m^3}$ oz. 
$\rho = (2770 \ \pm \ 20) \ \frac{kg}{m^3}$.
\\\\
S tega bi lahko sklepali, da je telo verjetno iz zlitine, ki vsebuje veliko aluminija, saj je njegova gostota: 
$\rho_{Al} = 2710 \ \frac{kg}{m^3}$.


\pagebreak

\section{Gostota kapljevine}
\subsection{Naloga 2}
\textbf{a)} Z menzuro in tehtnico izmerite gostoto 20 \% raztopine kuhinjske soli v vodi. Gostoto izmerite tudi z areometrom.\\\\
\textbf{b)} Z areometrom izmerite gostoto etilnega alkohola.

\subsection{Sistematične napake merilnikov}
Napaka areometra za raztopino NaCl: $\bm{0.01 \ \left[ \frac{g}{mL} \right]}$ \\
Napaka areometra za etilni alkohol: $\bm{0.005 \ \left[ \frac{g}{mL} \right]}$ \\
Merilno območje termometra: od $\bm{-199.9\ ^{\circ}C$} do \bm{$199.9\ ^{\circ}C}$ \\
Napaka tehtnice: \bm{1\ g} \\
Napaka menzure: \bm{2\ mL}
\\

\pagebreak

\subsection{Postopek in meritve}

Najprej smo pripravili 20 \% raztopino NaCl. Skupna masa raztopine je bila:
\begin{equation}
  m = (620 \ \pm \ 1 )\ g 
\end{equation}
\\
Nato smo izmerili volumen naše raztopine. Ker je menzura bila premajhna za 
celotno meritev volumna, smo to morali narediti trikrat.

\begin{equation}
  \label{eq:1}
  \begin{gathered}
    V = (250 \ mL \ \pm \ 2 \ mL) \ + \ (250 \ mL \ \pm \ 2 \ mL) + (51 \ mL \ \pm \ 2 \ mL) \\
    V = (553 \ \pm \ 6) \ mL \\
    V = 553 \cdot (1 \ \pm \ 0.01) \ mL
  \end{gathered}
\end{equation}
\\

Nato smo gostoto raztopine NaCl izmerili še z areometrom.

% Tabela za meritev gostote raztopine NaCl
\begin{table}[H]
  \centering
  \caption{Meritve gostote raztopine NaCl z areometrom}
  \begin{adjustwidth}{-2cm}{0cm}
  \begin{tabular}{cccccccc}
  \toprule
  Meritev & $\rho_{izm} \left[ \frac{g}{mL} \right]$ & $\overline{\rho}\ \left[ \frac{g}{mL} \right]$ & $\rho_{izm} - \overline{\rho}\ \left[ \frac{g}{mL} \right]$ & $\Delta \rho_{sist}\ \left[ \frac{g}{mL} \right]$ & $\Delta \rho_{sl}\ \left[ \frac{g}{mL} \right]$ & $\rho\ \left[ \frac{g}{mL} \right]$ & $T \ [^{\circ}C] $\\
  \midrule
  1 & 1.14 & \multirow{5}{*}{1.14} & 0 & \multirow{5}{*}{0.01} & \multirow{5}{*}{0} & & \multirow{5}{*}{19.6}\\
  2 & 1.14 &  & 0 & & & 1.14 \ \pm \ 0.01 \\
  3 & 1.15 &  & \sout{0.01} & & & = \\
  4 & 1.14 &  & 0 & & & 1.14 \cdot (1 \ \pm \ 0.01)\\
  5 & 1.15 &  & \sout{0.01} & & & \\
  % Add more rows here
  \bottomrule
  \end{tabular}
  \end{adjustwidth}
\end{table}

Nato smo še opravili meritve gostote etilnega alkohola, s pomočjo areometra.

\begin{table}[H]
  \centering
  \caption{Meritve gostote etilnega alkohola}
  \begin{adjustwidth}{-2cm}{0cm}
  \begin{tabular}{cccccccc}
  \toprule
  Meritev & $\rho_{izm} \left[ \frac{g}{mL} \right]$ & $\overline{\rho}\ \left[ \frac{g}{mL} \right]$ & $\rho_{izm} - \overline{\rho}\ \left[ \frac{g}{mL} \right]$ & $\Delta \rho_{sist}\ \left[ \frac{g}{mL} \right]$ & $\Delta \rho_{sl}\ \left[ \frac{g}{mL} \right]$ & $\rho\ \left[ \frac{g}{mL} \right]$ & $T \ [^{\circ}C] $\\
  \midrule
  1 & 0.805 & \multirow{5}{*}{0.805} & 0 & \multirow{5}{*}{0.005} & \multirow{5}{*}{0} & & \multirow{5}{*}{21.5} \\
  2 & 0.805 &  & 0 & & & 0.805 \ \pm \ 0.005 \\
  3 & 0.805 &  & 0 & & & = \\
  4 & 0.805 &  & 0 & & & 0.805 \cdot (1 \ \pm \ 0.006)\\
  5 & 0.805 &  & 0 & & & \\
  % Add more rows here
  \bottomrule
  \end{tabular}
  \end{adjustwidth}
\end{table}

\pagebreak

\subsection{Računanje gostote}

Računanje gostote 20 \% raztopine NaCl s pomočjo mase in volumna:

\begin{equation}
  \label{eq:1}
  \begin{gathered}
    \rho = \frac{620 \cdot (1 \ \pm \ 0.002)\ g}{553 \cdot (1 \pm 0.01) \ mL} \\
    \rho = 1.12 \cdot (1 \ \pm \ 0.01) \ \frac{g}{mL} \\
    \rho = 1120 \cdot (1 \ \pm \ 0.01) \ \frac{kg}{m^3}
  \end{gathered}
\end{equation}

\subsection{Rezultati}

Prišli smo do rezultatov, da je gostota 


\pagebreak
\section{Vprašanja}
\textbf{a)} Razložite, kako temperatura vpliva na merjenje gostote 
kapljevine. Za koliko odstotkov se spremeni gostota vode, če se 
temperatura spremeni za \bm{$1 \ K$}? Temperaturni koeficient prostorninskega 
razteska vode je \bm{$2.06 \cdot 10^{-6}K^{-1}$.}
\\\\
Relativno povečanje volumna vode je sorazmerno spremembi temperature:
\begin{equation}
  \frac{\Delta V}{V} = \beta \Delta T
\end{equation}
Če torej v enačbo vstavimo podatke dobimo:
\begin{equation}
  \label{eq:1}
  \begin{gathered}
    \frac{\Delta V}{V} = 2.06 \cdot 10^{-6}K^{-1} \cdot 1 \ K \\
    \frac{\Delta V}{V} = 2.06 \cdot 10^{-4} \%
  \end{gathered}
\end{equation}
\\

\noindent Torej bi se voda razteznila za $2.06 \cdot 10^{-4} \%$.

\bigskip


\noindent
\textbf{b)} Razložite fizikalni princip meritve gostote tekočin z areometrom. \\

\noindent
Areometer deluje na podlagi vzgona tekočine, v katero je potopljen.
Ima obliko ozke cevi z utežjo na dnu, ki je dovolj težka, da se areometer potopi v tekočino. 
Ko je areometer potopljen v tekočino, se potopi do ravni, kjer je vzgon enak teži areometra. 
Ta raven potopljenosti je odvisna od gostote tekočine.

\chapter{Vaja 2: Merjenje sile}

\section{Naloga}

\textbf{a)} Izvedite eksperiment, pri katerem boste merili silo v vzmeti in raztezek vzmeti. 
Izmerjene podatke prikažite z odvisnostjo sile vzmeti od raztezka. 
Iz diagrama določite prožnostni koeficient vzmeti.
\\\\
\textbf{b)} Sestavite vzmetno nihalo in izmerite silo v vzmeti v odvisnosti od časa. 
Meritev izvedite z računalniškim merilnim sistemom Vernier. 
Meritev opravite s tremi različnimi utežmi. 
Iz diagrama sile v odvisnosti od časa v vseh treh primerih odčitajte nihajni čas nihala in v vsakem primeru izračunajte konstanto vzmeti.

\section{Sistematične napake merilnikov}

Napaka ravnila: $\bm{0.01 \ cm}$ \\
Napaka tehtnice: $\bm{0.1 \ g}$ \\
Napaka za čas: $\frac{0.01}{15} \ s = \bm{0.001 \ s}$

\pagebreak

\section{Merjenje}

Za določanje koeficienta vzmeti sem najprej izmeril raztezke pri različnih masah uteži.

\begin{table}[H]
  \centering
  \caption{Raztezek vzmeti pri določeni teži}
  \begin{adjustwidth}{0cm}{0cm}
  \begin{tabular}{ccccccccc}
  \toprule
  Meritev & $m \ [g]$ & $\Delta m \ [g]$ & $ F \ [N] $ & $\Delta F \ [N]$ & $ x \ [cm] $ & $\Delta \ x \ [cm]$  \\
  \midrule
  1 & 50.9 & \multirow{7}{*}{0.1} & 0.499 & \multirow{7}{*}{0.001} & 6.8 & \multirow{7}{*}{0.1} & \\
  2 & 100.9 &  & 0.990 & & 13.5 & \\
  3 & 148.9 &  & 1.461 & & 19.8 &  \\
  4 & 199.4 &  & 1.956 & & 26.6 & \\
  5 & 148.9 &  & 1.461 & & 19.8 & \\
  6 & 100.9 &  & 0.990 & & 13.4 & \\
  7 & 50.9 &  & 0.499 & & 6.6 & \\
  % Add more rows here
  \bottomrule
  \end{tabular}
  \end{adjustwidth}
\end{table}

\noindent
Nato sem za uteži z masami $50.9  \ g$, $100.9 \ g$ in $148.9 \ g$ izmeril čas
nihanja za 15 nihajev.  


\begin{table}[H]
  \centering
  \caption{Nihajni čas}
  \begin{adjustwidth}{-2cm}{0cm}
  \begin{tabular}{cccccccccc}
  \toprule
  Meritev & $m \ [g]$ & $t_{izm} [s]$ & $t_{en \ nihaj} [s]$ &$ \overline{t} \ [s]$ & $t_{en \ nihaj} \ - \ \overline{t} \ [s]$ & $ \sigma \ [s] $ & $\Delta \ t_{sl} \ [s]$ & $ \Delta \ t_{sist} $ & $t \ [s]$\\
  \midrule
  1 & \multirow{5}{*}{50.9} & 8.26 & 0.551 & \multirow{5}{*}{0.551} & 0 & \multirow{5}{*}{0} & \multirow{5}{*}{0} & \multirow{5}{*}{0.001} &  \\
  2 &  & 8.26 & 0.551 & & 0 & & & & $0.551 \ \pm \ 0.001$\\
  3 &  & 8.24 & 0.549 & & \sout{-0.001} & & & & = \\
  4 &  & 8.26 & 0.551 & & 0 & & & & $0.551 \cdot (1 \ \pm \ 0.002)$ \\
  5 &  & 8.28 & 0.552 & & \sout{0.001} & \\
  \midrule
  1 & 50.9 & \multirow{7}{*}{0.1} & 0.499 & \multirow{7}{*}{0.001} & 6.8 & \multirow{7}{*}{0.1} & \\
  2 & 100.9 &  & 0.990 & & 13.5 & \\
  3 & 148.9 &  & 1.461 & & 19.8 &  \\
  4 & 199.4 &  & 1.956 & & 26.6 & \\
  5 & 148.9 &  & 1.461 & & 19.8 & \\
  \midrule
  1 & 50.9 & \multirow{7}{*}{0.1} & 0.499 & \multirow{7}{*}{0.001} & 6.8 & \multirow{7}{*}{0.1} & \\
  2 & 100.9 &  & 0.990 & & 13.5 & \\
  3 & 148.9 &  & 1.461 & & 19.8 &  \\
  4 & 199.4 &  & 1.956 & & 26.6 & \\
  5 & 148.9 &  & 1.461 & & 19.8 & \\


  % Add more rows here
  \bottomrule
  \end{tabular}
  \end{adjustwidth}
\end{table}

\section{Rezultati}

\begin{figure}[H]
  \caption{Graf F(x)}
  \includegraphics[width=\textwidth]{graf}
\end{figure}

\chapter{Vaja 3}
% Your content for section 1 here


\chapter{Vaja 4: Merjenje frekvence}

\section{Naloga}
Izmerite frekvenco vrtenja plošče, ki je pritrjena na elektromotor na dva načina:\\\\
\textbf{a)} z elektronskim merilnikom frekvence,\\
\textbf{b)} z modelom merilnika frekvence.\\\\
Primerjajte rezultata obeh meritev pri različnih frekvencah vrtenja plošče.\\\\
Te meritve sem opravil pri napetostih: \textbf{5 V, 6 V, 7 V, 9 V} in \textbf{12 V}, za vsako napetost 5-krat.
\afterpage{
\section{Meritve}

\begin{table}[H]

\caption{Merjenje frekvence uporabo elektronskega merilnika frekvence}
\begin{adjustwidth}{-3.5 cm}{0 cm}
\begin{tabular}{cccccccccccc}
\toprule
Meritev & Napetost & $\nu_{izm}$~[min\textsuperscript{-1}] & $\overline\nu$ & $\nu_{izm}$ - $\overline\nu$~[min\textsuperscript{-1}] &$\Delta\nu_{sist}$~[min\textsuperscript{-1}] & \sigma &$\Delta\nu_{slu}$ [min\textsuperscript{-1}] & \nu~[Hz]\\

%pri 5V

\midrule
1 & \multirow{5}{*}{5.0 V} & 654.4 & \multirow{5}{*}{658.9} & -4.5 & \multirow{5}{*}{0.1} & \multirow{5}{*}{6.5} & \multirow{5}{*}{2.9} & \\
2 & & 670.5 & & \sout{11.6}  & & & & $10.98 \ \pm \ 0.05$\\
3 & & 665.3 & & 6.4 & & & & =\\
4 & & 657.0 & & -1.9 & & & & $ 10.98 \cdot (1 \ \pm \ 0.005) $\\
5 & & 647.4 & & \sout{-11.5} &\\

%pri 6V

\midrule
6 & \multirow{5}{*}{6.0 V} & 1058 &  \multirow{5}{*}{1053} & 5 & \multirow{5}{*}{1} & \multirow{5}{*}{6} & \multirow{5}{*}{3} & \\
7 & & 1054 & & 1 & & & & $17.55 \ \pm \ 0.07$\\
8 & & 1037 & & \sout{-16} & & & & =\\
9 & & 1053 & & 0 & & & & $17.55 \cdot (1 \ \pm \ 0.004)$\\
10 & & 1062 & & \sout{9} & \\

%pri 7V

\midrule
11 & \multirow{5}{*}{7.0 V} & 1576 & \multirow{5}{*}{1563} & \sout{13} & \multirow{5}{*}{1} & \multirow{5}{*}{13}  & \multirow{5}{*}{6} & \\
12 & & 1575 & & 12 & & & & $ 26.05 \ \pm \ 0.12 $\\
13 & & 1532 & & \sout{-31} & & & & =\\
14 & & 1567 & & 4 & & & & $26.05 \cdot (1 \ \pm \ 0.005)$\\
15 & & 1565 & & 2 &\\

%pri 9 V

\midrule
16 & \multirow{5}{*}{9.0 V} & 2351 & \multirow{5}{*}{2413} & \sout{-62} & \multirow{5}{*}{1} & \multirow{5}{*}{57} & \multirow{5}{*}{25} & \\
17 & & 2354 & & \sout{-59} & & & & $ 40.22 \ \pm \ 0.22 $\\
18 & & 2449 & & 36 & & & & =\\
19 & & 2469 & & 56 & & & & $40.22 \cdot (1 \ \pm \ 0.005)  $\\
20 & & 2444 & & 31 &\\

%pri 12 V

\midrule
21 & \multirow{5}{*}{12.0 V} & 3917 & \multirow{5}{*}{3937} & -20 & \multirow{5}{*}{1} & \multirow{5}{*}{27} & \multirow{5}{*}{12} & \\
22 & & 3972 & & \sout{35} & & & & $65.61 \ \pm \ 0.45$\\
23 & & 3963 & & 26 & & & & =\\
24 & & 3905 & & \sout{-32} & & & & $65.61 \cdot (1 \ \pm \ 0.007) $\\
25 & & 3926 & & -11\\
% Add more rows here
\bottomrule
\end{tabular}
\end{adjustwidth}
\end{table}
}
\pagebreak





%% druga tabela %%%%

\begin{table}[H]

\caption{Mejrenje frekvence z uporabo osciloskopa}
\begin{adjustwidth}{-2.5 cm}{0 cm}
\begin{tabular}{cccccccccccc}
\toprule
Meritev & Napetost & $t_{izm} [s]$ & $\overline{t} \ [s]$ & $t_{izm}$ - $\overline{t} \ [s]$ & $t_{sist} \ [s]$ & \sigma & $\Delta t_{slu} [s]$ & \nu~[Hz]\\

%pri 5V

\midrule
1 & \multirow{5}{*}{5.0 V} & -0.080 & \multirow{5}{*}{0.088} & \sout{-0.008} & \multirow{5}{*}{0.004} & \multirow{5}{*}{0} & \multirow{5}{*}{0} & \\
2 & & 0.096 & & \sout{0.008}  & & & & $11 \cdot (1\ \pm \ 0.05)$\\
3 & & 0.088 & & 0 & & & & = \\
4 & & 0.088 & & 0 & & & & $11 \ \pm \ 1$\\
5 & & 0.088 & & 0 &\\

%pri 6V

\midrule
6 & \multirow{5}{*}{6.0 V} & 0.052 &  \multirow{5}{*}{0.052} & 0 & \multirow{5}{*}{0.002} & \multirow{5}{*}{0} & \multirow{5}{*}{0} &\\
7 & & 0.052 & & 0 & & & & $19 \cdot (1\ \pm \ 0.05)$\\
8 & & 0.050 & & \sout{-0.02} & & & & =\\
9 & & 0.052 & & 0 & & & & $19 \ \pm \ 1 $\\
10 & & 0.052 & & \sout{0} & \\

%pri 7V

\midrule
11 & \multirow{5}{*}{7.0 V} & 0.036 & \multirow{5}{*}{0.038} & \sout{-0.002} & \multirow{5}{*}{0.001} & \multirow{5}{*}{0} & \multirow{5}{*}{0} & \\
12 & & 0.038 & & 0 & & & & $26 \cdot (1 \ \pm \ 0.03) $\\
13 & & 0.038 & & 0 & & & & =\\
14 & & 0.039 & & \sout{0.001} & & & & $26 \ \pm \ 1$\\
15 & & 0.038 & & 0 &\\

%pri 9 V

\midrule
16 & \multirow{5}{*}{9.0 V} & 0.024 & \multirow{5}{*}{0.023} & \sout{0.001} & \multirow{5}{*}{0.001} & \multirow{5}{*}{0} & \multirow{5}{*}{0}\\
17 & & 0.023 & & 0 & & & & $43 \cdot (1 \ \pm \ 0.04)$\\
18 & & 0.022 & & \sout{-0.001} & & & & =\\
19 & & 0.023 & & 0 & & & & $43 \ \pm \ 2$\\
20 & & 0.023 & & 0 &\\

%pri 12 V

\midrule
21 & \multirow{5}{*}{12.0 V} & 0.0152 & \multirow{5}{*}{0.0150} & 0.0002 & \multirow{5}{*}{0.0004} & \multirow{5}{*}{0.0002} & \multirow{5}{*}{0.0001}\\
22 & & 0.0152 & & \sout{0.0002} & & & & $66.7 \cdot (1 \ \pm \ 0.03)$\\
23 & & 0.0148 & & -0.0002 & & & & =\\
24 & & 0.0148 & & \sout{-0.0002} & & & & $66.7 \ \pm \ 2.0 $\\
25 & & 0.0148 & & -0.0002\\
% Add more rows here
\bottomrule
\end{tabular}
\end{adjustwidth}
\end{table}
  








\end{document}
